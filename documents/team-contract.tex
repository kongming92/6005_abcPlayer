\documentclass[12pt]{article}

%% Include packages
%%\usepackage{times}
\usepackage[margin=1in,letterpaper,portrait]{geometry}
\usepackage{amsmath}
\usepackage{amssymb}
\usepackage{amsthm}
\usepackage{fancyhdr}
\usepackage[pdftex]{graphicx}
\usepackage{listings}
\usepackage{eucal}

%% Modify these variables
\newcommand{\student}{Alex Chernyakhovsky, Charles Liu, Lauren Stephens}
\newcommand{\studentemail}{achernya, cliu2014, lhs}
\newcommand{\course}{6.005}
\newcommand{\pset}{Project 1 Team Contract}

%% DO NOT MODIFY THIS SECTION

\fancyhf{}
\lhead{\course \\ \pset}
\rhead{\student \\ \studentemail}
\cfoot{\thepage}
\addtolength{\headheight}{30pt}
\renewcommand{\headrulewidth}{0.4pt}
\renewcommand{\footrulewidth}{0.4pt}

\pagestyle{fancy}

\newenvironment{problemset}{\begin{itemize}}{\end{itemize}}
\newenvironment{problem}[1]{\item #1\\}{}

%% END DO NOT MODIFY THIS SECTION

%% PUT MACROS HERE
\newcommand{\bra}{\left\langle}
\newcommand{\ket}{\right\rangle}
%%

\begin{document}

% Set up document title
\title{\pset}
\author{\student}
\date{}
\maketitle
\thispagestyle{fancy}

\section{Goals}
\begin{itemize}
\item What are the goals of the team? 

  The goals of the team are three-fold: the first is to attain an A on
  the assignment.  The second is to create a working, tested abc
  player that is simple to use (and possibly can be shown off to
  non-MIT friends who should be duly impressed).  The third goal is to
  work together and have fun as a team.  While we are interested in
  the end product, we also want to ensure that everybody contributed
  and had a good attitude.  We want the experience a positive one.

\item What are your personal goals for this assignment?

  \begin{itemize}
  \item Lauren Stephens: \\
    I second the team goals, and want to add that I will make an
    emphasis to pull my weight on the team because while I have not
    worked with Charles much, I have worked with Alex and understand
    he is currently more technically advanced than I am.
  \item Alex Chernyakhovsky: \\
    In addition to the team goals, I look forward to working with
    Lauren and Charles, in which I hope to work on my technical and
    non-technical communication skills, to allow the team to work
    coherently.  I also am interested in working on the design of the
    system itself, focusing on modularization and abstraction, that
    will allow the team to work independently initially, and then
    focus on the final integration as we approach completion.
  \item Charles Liu: \\
    In addition to all the team goals, one of my goals is to develop
    my abilities at working on a software project in a team. I'd like
    to be a good team member by carrying my own weight, providing good
    input to the overall picture while knowing when other people have
    better ideas, and ensuring that the finished project is of high
    quality. I'd like to take advantage of the opportunity to learn
    how to code better by learning from Lauren and Alex. Finally, this
    project relates to something that interests me, so I'd like to
    have fun with it!
  \end{itemize}

\item What kind of obstacles might you encounter in reaching your
  goals?

  The first obstacle is in design determination.  We want to ensure
  that each person agrees with and supports the overall design
  decisions and do not want to spend time trying to implement two
  different solutions or continually arguing about which architecture
  to use.  We will use a semi-democratic system to determine between
  conflicting designs presented by team members.  If two (or three)
  team members cannot agree on design patterns that bridge more than
  one person's work, we will first vote on which design pattern to
  use.  If the losing party is unhappy with the result, he or she has
  the right to ask for outside moderation.  In this case we will go to
  our TA and make our respective cases for each design pattern and let
  our TA help us decide.

  The second obstacle is in actual implementation.  We will assign the
  implementation of code to individuals however each individual should
  feel free to seek help from other group members.  If the whole group
  cannot figure out a particular problem that arises, we once again
  turn to our magic fixer: the TA.

  The third obstacle is time.  As a group we will set deadlines for
  having pieces of the work done.  If an individual cannot meet a
  deadline, they will be expected to inform the group early enough
  that we can decide what needs to be done and how to reassign work if
  necessary.  If a group member is consistently late we may resort to
  the procedures for one team member doing less work, but not without
  ample warnings.

\item What happens if all of you decide you want to get an A grade,
  but because of time constraints, one person decides that a B will be
  acceptable?

  In this case, a meeting will be held to determine if the remaining
  team members still want to achieve the original goals.  If they do
  and the third person is not contributing towards these goals, see
  next section for procedures.

\item Is it acceptable for one or two team members to do more work
  than the others in order to get the team an A?

  It is not acceptable for it to be excessive (work is very hard to
  divide exactly evenly).  It is acceptable for team members to
  specialize in what they do best, however everyone should try to work
  on at least a little bit of each stage.  It is also acceptable for
  some team members to have to email, nag, badger, and peer pressure
  another team member to do their part should a team member decide
  they do not want to do their work.  Ultimately if all social
  pressure fails or a team member has extenuating circumstances that
  prevent them from working on the project (my parents died, I am
  sick, etc.) the team will meet with a TA to discuss this.  If no
  resolution can be made that is satisfactory to all team members, a
  note will be attached to the final project indicating the respective
  amounts of work done.  It will ideally be signed by all team
  members.  Also the reflections section of the project will contain
  similar indications.  Clearly work is hard to divide exactly evenly,
  but we want to make sure that it is not overly unfair.
\end{itemize}

\section{Meeting Norms}
\begin{itemize}
\item Do you have a preference for when meetings will be held? Do you
  have a preference for where they should be held?

  \begin{itemize}
  \item Lauren:\\
    I prefer meetings on the sixth floor of Simmons (Alex and I live
    next to each other) however would be willing to accommodate
    meetings elsewhere on the MIT campus.  I have commitments to crew
    and classes and prefer the meeting not be held during crew
    practice (5-7) or class hours (with the exception of 6.005 time)
    unless I have given prior approval.
  \item Alex:\\
    I agree with Lauren that the sixth floor of Simmons is very
    convenient, and would prefer meetings there.  However, I recognize
    that Charles may have other obligations that make Simmons an
    inappropriate meeting place, and suggest the fifth floor of the
    Student Center, either in the Athena Cluster or SIPB Office as an
    alternative location. Other than the 6.005 class time, I cannot
    meet before 5pm.
  \end{itemize}

\item How will you use the in class time?

  We may use the in class time for team meetings as it would allow us
  to meet together with minimal schedule disruption.  If we are not
  meeting as a team, individual members are free to use the time as
  they see fit as long as they have fulfilled their obligations to the
  group

\item How often do you think the team will need to meet outside of
  class? How long do you anticipate meetings will be?

  As a team we will meet as often as necessary but hopefully at least
  every other day.  The meetings will probably be about 30 minutes of
  actual work and about 30 minutes of small-chat and bubble tea and
  hanging around. Obviously, team members may skip the unimportant 30
  minutes if they need to.

\item Will it be okay for team members to eat during meetings?

  Yes, in fact Lauren is expecting bubble tea at all meetings held in
  Simmons (thanks Alex!)

\item How will you record and distribute the minutes and action lists
  produced by each meeting?

  Lauren has volunteered to be the secretary and will record the
  minutes and actions lists in a Google document with the help of Alex
  and Charles.  This Google document will be shared among all team
  members to remind them of what went on in the meeting.

\end{itemize}

\section{Work Norms}
\begin{itemize}
\item How much time per week do you anticipate it will take to make
  the project successful?

  We expect the project to take about 15 hours a week per team member.
  This is a rough estimate and may vary on skill, how hard the project
  actually is, and how much time is spent actually working.

\item How will work be distributed?

  The work will be distributed first on a volunteer basis.  In a
  perfect world people will volunteer for their fair share and we
  don't have to worry.  In absence of a volunteer for a particular
  part, we will have a rotating system with the order based on a dice
  roll.  If someone does not seem to be doing a fair amount of work,
  the issue will be raised and suggestions for a better distribution
  will be voted on democratically finally we have magic TA moderation.

\item How will deadlines be set?

  Deadlines will be set both by the individual doing the work (``I can
  get it done by...'') and by the group (``we need it done by...'')
  If no resolution can be reached, we will try to put two people on
  the assignment or offer help.

\item How will you decide who should do which tasks?

  Volunteer, rotating, democratic, magic TA.  If volunteering fails we
  will use the rotating system.  If the rotating system fails or work
  is not being distributed evenly, we will use a democratic system.
  If that fails we get our magic TA.  Also testing, implementing, and
  reviewing will not be done by the same person for a given method.

\item Where will you record who is responsible for which tasks?

  Recorded on a shared Google Docs.

\item What will happen if someone does not follow through on a
  commitment (e.g., missing a deadline, not showing up to meetings)?

  We will first try to resolve the dispute internally so that all
  group members are satisfied work is being done equally and mostly
  equally well.  If this fails we will try to offer a compromise
  "make-up work" (i.e. you missed a meeting, you take notes at the
  next; you missed a deadline you agree to do more work for the next
  deadline).  If all else fails resort to procedures for one team
  member doing less work.

\item How will the work be reviewed?

  For each method we will split up the implementation, testing, and
  review.  The implementer will write the code and state that it is
  good (possibly performing their own simple checks).  The tester will
  write full tests for the method such that if the method passes their
  test, they are satisfied the method works.  Finally the reviewer
  will read over both the code and the tests to confirm.

\item What happens if people have different opinions on the quality of
  the work?

  Try to resolve internally be debating the quality of the work.  If
  no consensus can be reached take a democratic vote.  If all this
  fails, then bring in the magic TA,

\item What will you do if one or more team members are not doing their
  share of the work?

  Team members may have to email, nag, badger, and peer pressure
  another team member to do their part should a team member decide
  they do not want to do their work.  Ultimately if all social
  pressure fails or a team member has extenuating circumstances that
  prevent them from working on the project (my parents died, I am
  sick, etc.) the team will meet with a TA to discuss this.  If no
  resolution can be made that is satisfactory to all team members, a
  note will be attached to the final project indicating the respective
  amounts of work done.  It will ideally be signed by all team
  members.  Also the reflections section of the project will contain
  similar indications.


\item How will you deal with different work habits of individual team
  members (e.g., some people like to get assignments done as early as
  possible; others like to work under the pressure of a deadline)?

  We expect team members to meet all group---set deadlines, other than
  that, they can work as they please.

\end{itemize}

\section{Decision Making}
\begin{itemize}
\item Do you need consensus
  (100\% approval of all team members) before making a decision?

  We need at least two team members to make a decision.  If the third
  team member believes that this is the absolutely wrong decision, the
  third team member has the right to demand magic TA moderation.  We
  hope this will not have to happen.

\item What will you do if one of you fixates on a particular idea?

  Again, if one person is extremely unhappy will the group decision,
  he or she can demand the group meet with the TA.  All group members
  will then abide by that decision unless all three unanimously
  disagree

\end{itemize}

\end{document}

