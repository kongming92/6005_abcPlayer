\documentclass[12pt]{article}

%% Include packages
%%\usepackage{times}
\usepackage[margin=1in,letterpaper,portrait]{geometry}
\usepackage{amsmath}
\usepackage{amssymb}
\usepackage{amsthm}
\usepackage{fancyhdr}
\usepackage[pdftex]{graphicx}
\usepackage{listings}
\usepackage{eucal}

%% Modify these variables
\newcommand{\student}{Alex Chernyakhovsky, Charles Liu, Lauren Stephens}
\newcommand{\studentemail}{achernya, cliu2014, lhs}
\newcommand{\course}{6.005}
\newcommand{\pset}{Project 1 Design}

%% DO NOT MODIFY THIS SECTION

\fancyhf{}
\lhead{\course \\ \pset}
\rhead{\student \\ \studentemail}
\cfoot{\thepage}
\addtolength{\headheight}{30pt}
\renewcommand{\headrulewidth}{0.4pt}
\renewcommand{\footrulewidth}{0.4pt}

\pagestyle{fancy}

\newenvironment{problemset}{\begin{itemize}}{\end{itemize}}
\newenvironment{problem}[1]{\item #1\\}{}

%% END DO NOT MODIFY THIS SECTION

%% PUT MACROS HERE
\newcommand{\bra}{\left\langle}
\newcommand{\ket}{\right\rangle}
%%

\begin{document}

% Set up document title
\title{\pset}
\author{\student}
\date{}
\maketitle
\thispagestyle{fancy}
\
\section{Grammar}
\begin{verbatim}
abc-tune ::= abc-header abc-music

abc-header ::= field-number comment* field-title other-fields* field-key
        
field-number ::= "X:" DIGIT+ end-of-line
field-title ::= "T:" text end-of-line
other-fields ::= field-composer | field-default-length | field-meter
  | field-tempo | field-voice | comment
field-composer ::= "C:" text end-of-line
field-default-length ::= "L:" note-length-strict end-of-line
field-meter ::= "M:" meter end-of-line
field-tempo ::= "Q:" tempo end-of-line
field-voice ::= "V:" text end-of-line
field-key ::= "K:" key end-of-line

key ::= keynote [mode-minor]
keynote ::= basenote [key-accidental]
key-accidental ::= "#" | "b"
mode-minor ::= "m"

meter ::= "C" | "C|" | meter-fraction
meter-fraction ::= DIGIT+ "/" DIGIT+

tempo ::= DIGIT+

;;;;;;;;;;;;;;;;;;;;;;;;;;;;;;;;;;;;;;;;;;;;;;;;;;;;;;;;;;;;;;;;;;

abc-music ::= abc-line+
abc-line ::= (element+ linefeed) | mid-tune-field | comment
element ::= note-element | tuplet-element | barline | nth-repeat | space

note-element ::= (note | multi-note)

// note is either a pitch or a rest
note ::= note-or-rest [note-length]
note-or-rest ::= pitch | rest
pitch ::= [accidental] basenote [octave]
octave ::= ("'"+) | (","+)
note-length ::= [DIGIT+] ["/" [DIGIT+]]
note-length-strict ::= DIGIT+ "/" DIGIT+

; "^" is sharp, "_" is flat, and "=" is neutral
accidental ::= "^" | "^^" | "_" | "__" | "="

basenote ::= "C" | "D" | "E" | "F" | "G" | "A" | "B"
        | "c" | "d" | "e" | "f" | "g" | "a" | "b"

rest ::= "z"

// tuplets
tuplet-element ::= tuplet-spec note-element+
tuplet-spec ::= "(" DIGIT

// chords
multi-note ::= "[" note+ "]"

barline ::= "|" | "||" | "[|" | "|]" | ":|" | "|:"
nth-repeat ::= "[1" | "[2"

; A voice field might reappear in the middle of a piece
; to indicate the change of a voice
mid-tune-field ::= field-voice

comment ::= "%" text linefeed
end-of-line ::= comment | linefeed
\end{verbatim}

\section{Data Types}
\subsection{Lexer}
The first stage of reading the abc file is converting the text data to
tokens.  This requires the creation of a \texttt{Token} class, which
will have two fields: type and an optional \texttt{String} data.  The
following types must exist:

\begin{itemize}
\item \texttt{FIELD}\\
  Each field will be given its own value in the enumeration, and the
  optional data will be filled in. Specifically, these fields exist:
  \begin{itemize}
  \item \texttt{FIELD\_INDEX\_NUMBER}
  \item \texttt{FIELD\_TITLE}
  \item \texttt{FIELD\_COMPOSER\_NAME}
  \item \texttt{FIELD\_DEFAULT\_LENGTH}
  \item \texttt{FIELD\_METER}
  \item \texttt{FIELD\_TEMPO}
  \item \texttt{FIELD\_KEY}
  \item \texttt{FIELD\_VOICE}
  \end{itemize}
\item \texttt{DIGIT}\\
  The digit token will store a single number in the optional data
  field.
\item \texttt{NOTE\_LETTER}\\
  A single note, which will store the actual pitch-letter in the
  optional data field.
\item \texttt{ACCIDENTAL}\\
  A sharp or a flat, with the type stored in the optional data field.
\item \texttt{BARLINE}\\
  Signifies the end of a measure, with the optional data field holding
  the variations.
\item \texttt{NTH\_REPEAT}\\
  Variations of the repeat will be stored in the optional data field.
\item \texttt{MULTINOTE}\\
  Indicates a sequence of notes is a member of a chord, or that the
  chord is over. Specifically,
  \begin{itemize}
  \item \texttt{BEGIN\_MULTINOTE}
  \item \texttt{END\_MULTINOTE}
  \end{itemize}
\item \texttt{OCTAVE}\\
  Modifier for the note, either up or down.
\item \texttt{FRACTION\_BAR}\\
  A literal fraction bar, ``/''.
\item \texttt{COMMENT}\\
  A comment.
\item \texttt{TUPLET}\\
  Indicates the beginning of a tuplet, with the optional data
  containing the size of the tuplet.
\end{itemize}

The Lexer considers the following characters as whitespace: `` ``,
``\\t'', ``\\n'', ``\\r''. All of these will be ignored.

\subsection{Parser}
These tokens will then be combined by the Parser into an Abstract
Syntax Tree. The root of the tree will be the \texttt{Music} class,
which has \texttt{Voice}s in it. Each \texttt{Voice} will contain
\texttt{MusicSequence}s; a \texttt{MusicSequence} is either a
\texttt{Bar} or \texttt{Repeat}. A \texttt{Repeat} will have 1 or more
\texttt{Bar}s. \texttt{Bar}s will contain \texttt{Accidental}s and
\texttt{Note}s.

\begin{align*}
  Music&::=Map<String, Voice>\\
  Voice&::=List<MusicSequence>\\
  MusicSequence&::=Bar() + Repeat()\\
  Repeat&::=List<Bar>\\
  Bar&::=List<MusicalElement>\\
  MusicalElementContainer&::=Bar()+Chord()+Tuplet()\\
  ContainerType&::=ENUM(BAR, CHORD, TUPLET, REPEAT, VOICE)\\
  MusicalElement&::=Accidental(pitch: Pitch, value: int) \\
  &+ Note(pitch: Pitch, len: MusicalLength(num: int, denom: int)\\
  &+ Chord() + Tuplet()\\
  Chord&::=List<Note>\\
  Tuplet&::=List<Note>\\
\end{align*}

\section{Traversing the Abstract Syntax Tree}
The AST will be traversed using the Visitor pattern. The Visitor will
go through each level of the tree, applying correct transformations,
until it reaches a \texttt{Note}, at which point the \texttt{Note}
will be added to the SequencePlayer.

\section{State Machines}
\subsection{Parser}
The parser needs to maintain knowledge about the current parsing of
repeats, so that it knows if the current bar is alone or a member of
the repeat, or even ending the repeat.

\subsection{TranslateToSequenceVisitor}
This Visitor must maintain the state of the accidentals in the
\texttt{Bar}. This state variable is cleared out whenever a new
\texttt{Bar} is entered. Any errors in the AST will be played, with
minor corrections applied (e.g., invalid chord lengths will be reduced
to the length of the first note).

\subsection{OptimalTicksPerQuarterNoteVisitor}
This Visitor calculates the Least Common Multiple of the denominators
of note length, thereby providing the optimal number of ticks.

\subsection{MusicalWellFormednessVisitor}
This Visitor checks the Abstract Syntax Tree and returns any errors
that it finds. It does not attempt to fix any errors.

\section{Snapshot Diagrams}
Snapshot diagrams are too large to include in this document. Please
see the diagram files in the same folder as this document.

\end{document}
